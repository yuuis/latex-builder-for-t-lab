%======================================================% 
%----- T-lab LaTeX Template for "Resume" 
% 
%--- Please read "README" by all means (sakura) 
%======================================================% 

%------------------------------------------------------% 
%- Documentclass & Basic setting 
%------------------------------------------------------% 

\documentclass[a4j,twocolumn]{jsarticle} 

% Import original package 
\usepackage{./conf/t-lab} 
\usepackage{./conf/style} 
%======================================================%
%----- T-lab LaTeX Configuration template
%======================================================%

%------------------------------------------------------%
%- Import package
%------------------------------------------------------%

\usepackage{ifthen}				% if
\usepackage{url}				% reference


%------------------------------------------------------%
%- if
%------------------------------------------------------%

\newif\ifPDF 					% 画像出力
\newif\ifSRC					% ソースコード


%------------------------------------------------------%
%- Conditional branch
%--- コンパイルしたい条件によって変えて
%------------------------------------------------------%

%- 画像出力
\PDFtrue						% 画像出力をPDFにするか
%\PDFfalse						% ┗ しない場合

%- ソースコードを使うか (※ README.md を読む事)
%\SRCtrue						% 使う場合は true に
\SRCfalse						% ┗ 使わない場合


%------------------------------------------------------%
%- Image output setting
%------------------------------------------------------%

\ifPDF
	% PDF image output
	\usepackage[dvipdfmx]{graphicx}
	\usepackage[dvipdfmx]{color}
	\usepackage[dvipdfmx]{colortbl}
\else
	% dviout image output
	\usepackage[dviout]{graphicx}
	\usepackage[dviout]{color}
	\usepackage[dviout]{colortbl}
\fi


%------------------------------------------------------%
%- listings
%------------------------------------------------------%

\ifSRC

	\usepackage{listings}
	\usepackage{jlisting}

	\lstset{
		classoffset=0,
		numbers={left},
		stepnumber={1},
		sensitive={true},
		frame={tRBl},
		framesep={5pt},
		frameround={fttt},
		rulesep = 2pt,
		showstringspaces={false},
		tabsize={2},
		breaklines=true,
		xleftmargin=5mm,
		xrightmargin=3mm,
		%framexleftmargin=6mm,							% 行番号をフレームに入れる
		basicstyle={\ttfamily \footnotesize},
		numberstyle={\scriptsize},
		stringstyle={\ttfamily \color[cmyk]{0,0.8,0,0}},
		commentstyle={\color[rgb]{0,0.4,0}},
		morecomment=[l]{;;},
		keywordstyle={\ttfamily \color[rgb]{0,0,1}}
	}

	%- Java
	\newenvironment{Java}[0]{
	\lstset{
		language={Java},
		classoffset=1,
		keywordstyle={\ttfamily \color[rgb]{1, 0, 0}},
		morekeywords={
			Louise,
		}
	}}{}

	%- Code paste
	\newcommand{\srcPst}[4]{
		\vspace{3mm}
			\begin{#1} \lstinputlisting[caption=#4, label=src:#3]{./src/#2} \end{#1}
		\vspace{5mm}
	}

	\newcommand{\srcref}[1]{{\bf \lstlistingname~\ref{src:#1}}}		% 参照
	\renewcommand{\lstlistingname}{{\bf リスト}}					% キャプション
	\renewcommand{\thelstnumber}{\arabic{lstnumber}:}			% 行番号の表示

\fi


%------------------------------------------------------%
%- Define original command
%------------------------------------------------------%

%- Reference
\newcommand{\figref}[1]{{\bf \figurename~\ref{fig:#1}}}
\newcommand{\tabref}[1]{{\bf \tablename~\ref{tab:#1}}}
\newcommand{\equref}[1]{{\bf 式~(\ref{equ:#1})}}

%- Quotation
\newcommand{\dq}[1]{`` #1 ''}
\newcommand{\bdq}[1]{{\bf \dq{#1}}}

%- Bibliography
\newcommand{\bib}[4]{\bibitem{#1} #2 : ``#3''{ }(#4).}
\newcommand{\bibURL}[5]{\bibitem{#1} #2 : ``#3''{ }\url{#4}{ }(#5).}

%- Image paste
\newcommand{\figPst}[3]{
	\vspace{3mm} \begin{figure}[tbh]
		\begin{center}
			\fbox{	\includegraphics[width=#1mm]{./figure/#2.png}	}
			\caption{#3}\label{fig:#2}
		\end{center}
	\end{figure} \vspace{-4mm}
}
 

% 英語で書こうと思う人は,以下をコメントアウトするとそれっぽいとこが英語になって良い 
% jsarticle -> article するとコンパイルできんなるので. 
% \renewcommand{\tablename}{Table.} 
% \renewcommand{\figurename}{Fig.} 
% \renewcommand{\refname}{References} 

% Define basic information 
% - ここら辺を空気読んで編集して 
\Author{石黒 優}
\aAuthor{Yu ISHIGURO}
\ID{C0116327}
\Title{Context Awarenessに基づくパーソナルサーバサービスの提案と実装!}
\Laboratory{田胡・松岡}
\HeaderSubject{2019年度 コンピュータサイエンス学部 中間発表}
\Date{2019年08月19日}


%------------------------------------------------------% 
%- Document 
%------------------------------------------------------% 

\begin{document} 
\twocolumn[\makeHeader]

\section{背景}
\subsection{webサービスにおけるユーザ最適化の現状}
現代の多くのwebサービスは, 利用者の大量のデータを集めてセグメント毎にレコメンドの最適化を行い, マーケティング効果を向上させている.
しかし, セグメントという単位での最適化では, その個人に対しての最適化がなされていないため, 確実に利用者の意図に沿うとは限らず, 利用者に忖度した結果とならない.

\subsection{個人情報の扱い}
利用者個人に対しての最適化は, 技術的には可能であるが, 個人情報の取り扱いが問題となり, 実現が困難である.
つまり, 利用者のデータを吸い上げてまとめて最適化するモデルは限界を迎えつつあり, 全く新しいモデルを考えなければならない段階にあると言える.

%------------------------------------------------------% 

\section{提案}
前節で述べた問題を解決する為, 「パーソナルサーバサービス(Personal Server Service : PSS)」を提案する.

\subsection{新しいユーザ最適化のモデル} 
現状のwebサービスで用いられているモデルは, 1つのAIに多くの利用者が接続し, AIは多人数に対しての学習をする.
私が本研究で提案する新しいモデルでは, 利用者1人につきその利用者専用のAIが1つ用意され, その利用者のために最適化されるというものである.

\subsection{パーソナルサーバサービス(以下PSS)の構成}
PSSの全体構成を, \figref{architecture}に示す.
\figPst{80}{architecture}{PSSの全体構成}
利用者は, スマートグラスやスマートフォンなどのモバイル端末, スマート家電をはじめとするIoT機器などの様々な媒体から個人データを送信する.
送信された個人データは, パートナーアシスタントによって収集され, 分析が行われる. パートナーアシスタントについては, 2.3.1節にて詳しく述べる.
収集された個人データは, 個人専用の個人データリポジトリに蓄積される. 個人データリポジトリについては, 2.3.2節にて詳しく述べる.
また, パートナーアシスタント, は利用者との自然会話から, 必要なサービスを判断し, レコメンドシステムに対してリクエストを行う.
リクエストを受けたレコメンドシステムは, リクエスト内容に応じるサービスを選別し, 個人データリポジトリにある個人データを基に, 利用者に最適化された検索条件や匿名データを生成し, サービス内容を受け取る.
レコメンドシステムは受け取ったサービス内容をパートナーアシスタントを介して利用者に提供する.
レコメンドシステムについては, 2.3.3節にて詳しく述べる.

\subsection{PSSの構成要素}
\subsubsection{パートナーアシスタント}
膨大な個人データにより学習されたインタラクティブマルチモーダルAIを内包し, 利用者専用のAIアシスタントを提供する.
専用AIによるアシスタントでは, 利用者の趣味嗜好などを学習し, 個人に最適化された学習結果を形成する.
PSS利用者が複数存在した場合でも, 利用者ごとのパートナーアシスタントは全く違った学習結果を持ち, それぞれが利用者に対して異なる情報を保持することで, 個人専用AIアシスタントの提供を行うことが可能となる.

\subsubsection{個人データリポジトリ}
クラウド上にコンテナ技術を用いて構築される, 他の利用者と混在することのない個人専用のデータリポジトリ.
利用者専用のあらゆる個人データを個人専用のリポジトリのみで収集・管理することにより, 利用者の同意を得ない形での個人データの流用や, 特定企業による個人データの独占を防ぐことができる.

\subsubsection{レコメンドシステム}
収集された個人データを用いて, 利用者に対し最適な提案を行う機能を提供する.
PSSでは, 利用者の同意のもとでプラグアブルに外部サービスを追加・拡張する. 
追加されたサービスからそのサービスに関する情報の取得した上で, 個人データに基づいて利用者にとって忖度された提案を導き出す.
外部サービス側には利用者を特定しない匿名データとして扱うことによって, 利用者の個人データが外部へ漏洩する事を防ぐことができる.

%------------------------------------------------------% 

\section{プロトタイピング}
\subsection{目的}
Context Awarenessに基づくレコメンドシステムを用いて, レコメンドを行う際に重要なコンテキストを検証する.

\subsection{プロトタイプ}
LINE Botと位置情報収集アプリケーションを用いたプロトタイプを作成した.
このプロトタイプは, 利用者の位置情報を取得し続け, その位置情報と利用者の趣味・嗜好情報を基に利用者に最適な飲食店情報をレコメンドする.

\subsection{プロトタイプの構成}
プロトタイプの全体構成を, \figref{proto_architecture}に示す.
\figPst{80}{proto_architecture}{プロトタイプの全体構成}

\subsection{プロトタイプの実行例}
利用者の検索履歴と位置情報, また時間帯を基にレコメンドを行なっている.
レコメンドの動作をしている画像を, \figref{recommend}に示す.
\figPst{35}{recommend}{レコメンドの動作画像}
レコメンド後, 利用者の要望を基に再度レコメンドしている.
また, 個人データリポジトリの予算傾向を1段低く更新し, 以降のレコメンドに反映させる.
ユーザのレコメンドを基に再度レコメンドしている画像を, \figref{re-recommend}に示す.
\figPst{35}{re-recommend}{再レコメンドの動作画像}

%------------------------------------------------------% 

\section{最終目標}
\subsection{本研究の目標}
本開発は最終的に, オープンソースによる公開を行うと同時にサービス事業者への売り込みを行い, 大規模なビジネスの展開を行うことを目標とする.


%------------------------------------------------------% 
%- References 
%------------------------------------------------------% 

\begin{thebibliography}{99}
\bibURL{gdpr}{EU一般データ保護規則(General Data Protection Regulation:GDPR)}{EUR-Lex}{http://eur-lex.europa.eu/legal-content/EN/TXT/?uri=uriserv:OJ.L\_.2016.119.01.0001.01.ENG\&toc=OJ:L:2016:119:TOC}{2019/07/08}
\bib{context_awareness_application}{上岡 英史 著}{コンテクストアウェアネスを用いたアプリケーションの研究動向}{IPSJ Magazine Vol.44 No.3 Mar. 2003, pp265-269}
\end{thebibliography} 

\end{document} 
