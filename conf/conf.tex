%======================================================%
%----- T-lab LaTeX Configuration template
%======================================================%

%------------------------------------------------------%
%- Import package
%------------------------------------------------------%

\usepackage{ifthen}				% if
\usepackage{url}				% reference


%------------------------------------------------------%
%- if
%------------------------------------------------------%

\newif\ifPDF 					% 画像出力
\newif\ifSRC					% ソースコード


%------------------------------------------------------%
%- Conditional branch
%--- コンパイルしたい条件によって変えて
%------------------------------------------------------%

%- 画像出力
\PDFtrue						% 画像出力をPDFにするか
%\PDFfalse						% ┗ しない場合

%- ソースコードを使うか (※ README.md を読む事)
%\SRCtrue						% 使う場合は true に
\SRCfalse						% ┗ 使わない場合


%------------------------------------------------------%
%- Image output setting
%------------------------------------------------------%

\ifPDF
	% PDF image output
	\usepackage[dvipdfmx]{graphicx}
	\usepackage[dvipdfmx]{color}
	\usepackage[dvipdfmx]{colortbl}
\else
	% dviout image output
	\usepackage[dviout]{graphicx}
	\usepackage[dviout]{color}
	\usepackage[dviout]{colortbl}
\fi


%------------------------------------------------------%
%- listings
%------------------------------------------------------%

\ifSRC

	\usepackage{listings}
	\usepackage{jlisting}

	\lstset{
		classoffset=0,
		numbers={left},
		stepnumber={1},
		sensitive={true},
		frame={tRBl},
		framesep={5pt},
		frameround={fttt},
		rulesep = 2pt,
		showstringspaces={false},
		tabsize={2},
		breaklines=true,
		xleftmargin=5mm,
		xrightmargin=3mm,
		%framexleftmargin=6mm,							% 行番号をフレームに入れる
		basicstyle={\ttfamily \footnotesize},
		numberstyle={\scriptsize},
		stringstyle={\ttfamily \color[cmyk]{0,0.8,0,0}},
		commentstyle={\color[rgb]{0,0.4,0}},
		morecomment=[l]{;;},
		keywordstyle={\ttfamily \color[rgb]{0,0,1}}
	}

	%- Java
	\newenvironment{Java}[0]{
	\lstset{
		language={Java},
		classoffset=1,
		keywordstyle={\ttfamily \color[rgb]{1, 0, 0}},
		morekeywords={
			Louise,
		}
	}}{}

	%- Code paste
	\newcommand{\srcPst}[4]{
		\vspace{3mm}
			\begin{#1} \lstinputlisting[caption=#4, label=src:#3]{./src/#2} \end{#1}
		\vspace{5mm}
	}

	\newcommand{\srcref}[1]{{\bf \lstlistingname~\ref{src:#1}}}		% 参照
	\renewcommand{\lstlistingname}{{\bf リスト}}					% キャプション
	\renewcommand{\thelstnumber}{\arabic{lstnumber}:}			% 行番号の表示

\fi


%------------------------------------------------------%
%- Define original command
%------------------------------------------------------%

%- Reference
\newcommand{\figref}[1]{{\bf \figurename~\ref{fig:#1}}}
\newcommand{\tabref}[1]{{\bf \tablename~\ref{tab:#1}}}
\newcommand{\equref}[1]{{\bf 式~(\ref{equ:#1})}}

%- Quotation
\newcommand{\dq}[1]{`` #1 ''}
\newcommand{\bdq}[1]{{\bf \dq{#1}}}

%- Bibliography
\newcommand{\bib}[4]{\bibitem{#1} #2 : ``#3''{ }(#4).}
\newcommand{\bibURL}[5]{\bibitem{#1} #2 : ``#3''{ }\url{#4}{ }(#5).}

%- Image paste
\newcommand{\figPst}[3]{
	\vspace{3mm} \begin{figure}[tbh]
		\begin{center}
			\fbox{	\includegraphics[width=#1mm]{./figure/#2.png}	}
			\caption{#3}\label{fig:#2}
		\end{center}
	\end{figure} \vspace{-4mm}
}

%- Image paste without column
\newcommand{\figPstNC}[3]{
	\vspace{3mm} \begin{figure*}[tbh]
		\begin{center}
			\fbox{	\includegraphics[width=#1mm]{./figure/#2.png}	}
			\caption{#3}\label{fig:#2}
		\end{center}
	\end{figure*} \vspace{-4mm}
}
